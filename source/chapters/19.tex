%!TeX root=../draculatop.tex
\chapter[Chapter \thechapter]{}

\section{Jonathan Harker's Journal}
	
\begin{diary}{1 October, 5 \textsc{a.m.}}
I went with the party to the search with an easy mind, for I think I never saw Mina so absolutely strong and well. I am so glad that she consented to hold back and let us men do the work. Somehow, it was a dread to me that she was in this fearful business at all; but now that her work is done, and that it is due to her energy and brains and foresight that the whole story is put together in such a way that every point tells, she may well feel that her part is finished, and that she can henceforth leave the rest to us. We were, I think, all a little upset by the scene with Mr Renfield. When we came away from his room we were silent till we got back to the study. Then Mr Morris said to Dr Seward:—

<Say, Jack, if that man wasn't attempting a bluff, he is about the sanest lunatic I ever saw. I'm not sure, but I believe that he had some serious purpose, and if he had, it was pretty rough on him not to get a chance.> Lord Godalming and I were silent, but Dr Van Helsing added:—

<Friend John, you know more of lunatics than I do, and I'm glad of it, for I fear that if it had been to me to decide I would before that last hysterical outburst have given him free. But we live and learn, and in our present task we must take no chance, as my friend Quincey would say. All is best as they are.> Dr Seward seemed to answer them both in a dreamy kind of way:—

<I don't know but that I agree with you. If that man had been an ordinary lunatic I would have taken my chance of trusting him; but he seems so mixed up with the Count in an indexy kind of way that I am afraid of doing anything wrong by helping his fads. I can't forget how he prayed with almost equal fervour for a cat, and then tried to tear my throat out with his teeth. Besides, he called the Count <lord and master,> and he may want to get out to help him in some diabolical way. That horrid thing has the wolves and the rats and his own kind to help him, so I suppose he isn't above trying to use a respectable lunatic. He certainly did seem earnest, though. I only hope we have done what is best. These things, in conjunction with the wild work we have in hand, help to unnerve a man.> The Professor stepped over, and laying his hand on his shoulder, said in his grave, kindly way:—

<Friend John, have no fear. We are trying to do our duty in a very sad and terrible case; we can only do as we deem best. What else have we to hope for, except the pity of the good God?> Lord Godalming had slipped away for a few minutes, but now he returned. He held up a little silver whistle, as he remarked:—

<That old place may be full of rats, and if so, I've got an antidote on call.> Having passed the wall, we took our way to the house, taking care to keep in the shadows of the trees on the lawn when the moonlight shone out. When we got to the porch the Professor opened his bag and took out a lot of things, which he laid on the step, sorting them into four little groups, evidently one for each. Then he spoke:—

<My friends, we are going into a terrible danger, and we need arms of many kinds. Our enemy is not merely spiritual. Remember that he has the strength of twenty men, and that, though our necks or our windpipes are of the common kind—and therefore breakable or crushable—his are not amenable to mere strength. A stronger man, or a body of men more strong in all than him, can at certain times hold him; but they cannot hurt him as we can be hurt by him. We must, therefore, guard ourselves from his touch. Keep this near your heart>—as he spoke he lifted a little silver crucifix and held it out to me, I being nearest to him—<put these flowers round your neck>—here he handed to me a wreath of withered garlic blossoms—<for other enemies more mundane, this revolver and this knife; and for aid in all, these so small electric lamps, which you can fasten to your breast; and for all, and above all at the last, this, which we must not desecrate needless.> This was a portion of Sacred Wafer, which he put in an envelope and handed to me. Each of the others was similarly equipped. <Now,> he said, <friend John, where are the skeleton keys? If so that we can open the door, we need not break house by the window, as before at Miss Lucy's.>

Dr Seward tried one or two skeleton keys, his mechanical dexterity as a surgeon standing him in good stead. Presently he got one to suit; after a little play back and forward the bolt yielded, and, with a rusty clang, shot back. We pressed on the door, the rusty hinges creaked, and it slowly opened. It was startlingly like the image conveyed to me in Dr Seward's diary of the opening of Miss Westenra's tomb; I fancy that the same idea seemed to strike the others, for with one accord they shrank back. The Professor was the first to move forward, and stepped into the open door.

<\textit{In manus tuas, Domine!}> he said, crossing himself as he passed over the threshold. We closed the door behind us, lest when we should have lit our lamps we should possibly attract attention from the road. The Professor carefully tried the lock, lest we might not be able to open it from within should we be in a hurry making our exit. Then we all lit our lamps and proceeded on our search.

The light from the tiny lamps fell in all sorts of odd forms, as the rays crossed each other, or the opacity of our bodies threw great shadows. I could not for my life get away from the feeling that there was some one else amongst us. I suppose it was the recollection, so powerfully brought home to me by the grim surroundings, of that terrible experience in Transylvania. I think the feeling was common to us all, for I noticed that the others kept looking over their shoulders at every sound and every new shadow, just as I felt myself doing.

The whole place was thick with dust. The floor was seemingly inches deep, except where there were recent footsteps, in which on holding down my lamp I could see marks of hobnails where the dust was cracked. The walls were fluffy and heavy with dust, and in the corners were masses of spider's webs, whereon the dust had gathered till they looked like old tattered rags as the weight had torn them partly down. On a table in the hall was a great bunch of keys, with a time-yellowed label on each. They had been used several times, for on the table were several similar rents in the blanket of dust, similar to that exposed when the Professor lifted them. He turned to me and said:—

<You know this place, Jonathan. You have copied maps of it, and you know it at least more than we do. Which is the way to the chapel?> I had an idea of its direction, though on my former visit I had not been able to get admission to it; so I led the way, and after a few wrong turnings found myself opposite a low, arched oaken door, ribbed with iron bands. <This is the spot,> said the Professor as he turned his lamp on a small map of the house, copied from the file of my original correspondence regarding the purchase. With a little trouble we found the key on the bunch and opened the door. We were prepared for some unpleasantness, for as we were opening the door a faint, malodorous air seemed to exhale through the gaps, but none of us ever expected such an odour as we encountered. None of the others had met the Count at all at close quarters, and when I had seen him he was either in the fasting stage of his existence in his rooms or, when he was gloated with fresh blood, in a ruined building open to the air; but here the place was small and close, and the long disuse had made the air stagnant and foul. There was an earthy smell, as of some dry miasma, which came through the fouler air. But as to the odour itself, how shall I describe it? It was not alone that it was composed of all the ills of mortality and with the pungent, acrid smell of blood, but it seemed as though corruption had become itself corrupt. Faugh! it sickens me to think of it. Every breath exhaled by that monster seemed to have clung to the place and intensified its loathsomeness.

Under ordinary circumstances such a stench would have brought our enterprise to an end; but this was no ordinary case, and the high and terrible purpose in which we were involved gave us a strength which rose above merely physical considerations. After the involuntary shrinking consequent on the first nauseous whiff, we one and all set about our work as though that loathsome place were a garden of roses.

We made an accurate examination of the place, the Professor saying as we began:—

<The first thing is to see how many of the boxes are left; we must then examine every hole and corner and cranny and see if we cannot get some clue as to what has become of the rest.> A glance was sufficient to show how many remained, for the great earth chests were bulky, and there was no mistaking them.

There were only twenty-nine left out of the fifty! Once I got a fright, for, seeing Lord Godalming suddenly turn and look out of the vaulted door into the dark passage beyond, I looked too, and for an instant my heart stood still. Somewhere, looking out from the shadow, I seemed to see the high lights of the Count's evil face, the ridge of the nose, the red eyes, the red lips, the awful pallor. It was only for a moment, for, as Lord Godalming said, <I thought I saw a face, but it was only the shadows,> and resumed his inquiry, I turned my lamp in the direction, and stepped into the passage. There was no sign of any one; and as there were no corners, no doors, no aperture of any kind, but only the solid walls of the passage, there could be no hiding-place even for \textit{him}. I took it that fear had helped imagination, and said nothing.

A few minutes later I saw Morris step suddenly back from a corner, which he was examining. We all followed his movements with our eyes, for undoubtedly some nervousness was growing on us, and we saw a whole mass of phosphorescence, which twinkled like stars. We all instinctively drew back. The whole place was becoming alive with rats.

For a moment or two we stood appalled, all save Lord Godalming, who was seemingly prepared for such an emergency. Rushing over to the great iron-bound oaken door, which Dr Seward had described from the outside, and which I had seen myself, he turned the key in the lock, drew the huge bolts, and swung the door open. Then, taking his little silver whistle from his pocket, he blew a low, shrill call. It was answered from behind Dr Seward's house by the yelping of dogs, and after about a minute three terriers came dashing round the corner of the house. Unconsciously we had all moved towards the door, and as we moved I noticed that the dust had been much disturbed: the boxes which had been taken out had been brought this way. But even in the minute that had elapsed the number of the rats had vastly increased. They seemed to swarm over the place all at once, till the lamplight, shining on their moving dark bodies and glittering, baleful eyes, made the place look like a bank of earth set with fireflies. The dogs dashed on, but at the threshold suddenly stopped and snarled, and then, simultaneously lifting their noses, began to howl in most lugubrious fashion. The rats were multiplying in thousands, and we moved out.

Lord Godalming lifted one of the dogs, and carrying him in, placed him on the floor. The instant his feet touched the ground he seemed to recover his courage, and rushed at his natural enemies. They fled before him so fast that before he had shaken the life out of a score, the other dogs, who had by now been lifted in the same manner, had but small prey ere the whole mass had vanished.

With their going it seemed as if some evil presence had departed, for the dogs frisked about and barked merrily as they made sudden darts at their prostrate foes, and turned them over and over and tossed them in the air with vicious shakes. We all seemed to find our spirits rise. Whether it was the purifying of the deadly atmosphere by the opening of the chapel door, or the relief which we experienced by finding ourselves in the open I know not; but most certainly the shadow of dread seemed to slip from us like a robe, and the occasion of our coming lost something of its grim significance, though we did not slacken a whit in our resolution. We closed the outer door and barred and locked it, and bringing the dogs with us, began our search of the house. We found nothing throughout except dust in extraordinary proportions, and all untouched save for my own footsteps when I had made my first visit. Never once did the dogs exhibit any symptom of uneasiness, and even when we returned to the chapel they frisked about as though they had been rabbit-hunting in a summer wood.

The morning was quickening in the east when we emerged from the front. Dr Van Helsing had taken the key of the hall-door from the bunch, and locked the door in orthodox fashion, putting the key into his pocket when he had done.

<So far,> he said, <our night has been eminently successful. No harm has come to us such as I feared might be and yet we have ascertained how many boxes are missing. More than all do I rejoice that this, our first—and perhaps our most difficult and dangerous—step has been accomplished without the bringing thereinto our most sweet Madam Mina or troubling her waking or sleeping thoughts with sights and sounds and smells of horror which she might never forget. One lesson, too, we have learned, if it be allowable to argue \textit{a particulari}: that the brute beasts which are to the Count's command are yet themselves not amenable to his spiritual power; for look, these rats that would come to his call, just as from his castle top he summon the wolves to your going and to that poor mother's cry, though they come to him, they run pell-mell from the so little dogs of my friend Arthur. We have other matters before us, other dangers, other fears; and that monster—he has not used his power over the brute world for the only or the last time to-night. So be it that he has gone elsewhere. Good! It has given us opportunity to cry <check> in some ways in this chess game, which we play for the stake of human souls. And now let us go home. The dawn is close at hand, and we have reason to be content with our first night's work. It may be ordained that we have many nights and days to follow, if full of peril; but we must go on, and from no danger shall we shrink.>

The house was silent when we got back, save for some poor creature who was screaming away in one of the distant wards, and a low, moaning sound from Renfield's room. The poor wretch was doubtless torturing himself, after the manner of the insane, with needless thoughts of pain.

I came tiptoe into our own room, and found Mina asleep, breathing so softly that I had to put my ear down to hear it. She looks paler than usual. I hope the meeting to-night has not upset her. I am truly thankful that she is to be left out of our future work, and even of our deliberations. It is too great a strain for a woman to bear. I did not think so at first, but I know better now. Therefore I am glad that it is settled. There may be things which would frighten her to hear; and yet to conceal them from her might be worse than to tell her if once she suspected that there was any concealment. Henceforth our work is to be a sealed book to her, till at least such time as we can tell her that all is finished, and the earth free from a monster of the nether world. I daresay it will be difficult to begin to keep silence after such confidence as ours; but I must be resolute, and to-morrow I shall keep dark over to-night's doings, and shall refuse to speak of anything that has happened. I rest on the sofa, so as not to disturb her.
\end{diary}
 

\begin{diary}{1 October, later.}
I suppose it was natural that we should have all overslept ourselves, for the day was a busy one, and the night had no rest at all. Even Mina must have felt its exhaustion, for though I slept till the sun was high, I was awake before her, and had to call two or three times before she awoke. Indeed, she was so sound asleep that for a few seconds she did not recognize me, but looked at me with a sort of blank terror, as one looks who has been waked out of a bad dream. She complained a little of being tired, and I let her rest till later in the day. We now know of twenty-one boxes having been removed, and if it be that several were taken in any of these removals we may be able to trace them all. Such will, of course, immensely simplify our labour, and the sooner the matter is attended to the better. I shall look up Thomas Snelling to-day.
	\end{diary}

\section{Dr Seward's Diary}

\begin{diary}{1 October.}
It was towards noon when I was awakened by the Professor walking into my room. He was more jolly and cheerful than usual, and it is quite evident that last night's work has helped to take some of the brooding weight off his mind. After going over the adventure of the night he suddenly said:—

<Your patient interests me much. May it be that with you I visit him this morning? Or if that you are too occupy, I can go alone if it may be. It is a new experience to me to find a lunatic who talk philosophy, and reason so sound.> I had some work to do which pressed, so I told him that if he would go alone I would be glad, as then I should not have to keep him waiting; so I called an attendant and gave him the necessary instructions. Before the Professor left the room I cautioned him against getting any false impression from my patient. <But,> he answered, <I want him to talk of himself and of his delusion as to consuming live things. He said to Madam Mina, as I see in your diary of yesterday, that he had once had such a belief. Why do you smile, friend John?>

<Excuse me,> I said, <but the answer is here.> I laid my hand on the type-written matter. <When our sane and learned lunatic made that very statement of how he \textit{used} to consume life, his mouth was actually nauseous with the flies and spiders which he had eaten just before Mrs Harker entered the room.> Van Helsing smiled in turn. <Good!> he said. <Your memory is true, friend John. I should have remembered. And yet it is this very obliquity of thought and memory which makes mental disease such a fascinating study. Perhaps I may gain more knowledge out of the folly of this madman than I shall from the teaching of the most wise. Who knows?> I went on with my work, and before long was through that in hand. It seemed that the time had been very short indeed, but there was Van Helsing back in the study. <Do I interrupt?> he asked politely as he stood at the door.

<Not at all,> I answered. <Come in. My work is finished, and I am free. I can go with you now, if you like.>

<It is needless; I have seen him!>

<Well?>

<I fear that he does not appraise me at much. Our interview was short. When I entered his room he was sitting on a stool in the centre, with his elbows on his knees, and his face was the picture of sullen discontent. I spoke to him as cheerfully as I could, and with such a measure of respect as I could assume. He made no reply whatever. <Don't you know me?> I asked. His answer was not reassuring: <I know you well enough; you are the old fool Van Helsing. I wish you would take yourself and your idiotic brain theories somewhere else. Damn all thick-headed Dutchmen!> Not a word more would he say, but sat in his implacable sullenness as indifferent to me as though I had not been in the room at all. Thus departed for this time my chance of much learning from this so clever lunatic; so I shall go, if I may, and cheer myself with a few happy words with that sweet soul Madam Mina. Friend John, it does rejoice me unspeakable that she is no more to be pained, no more to be worried with our terrible things. Though we shall much miss her help, it is better so.>

<I agree with you with all my heart,> I answered earnestly, for I did not want him to weaken in this matter. <Mrs Harker is better out of it. Things are quite bad enough for us, all men of the world, and who have been in many tight places in our time; but it is no place for a woman, and if she had remained in touch with the affair, it would in time infallibly have wrecked her.>

So Van Helsing has gone to confer with Mrs Harker and Harker; Quincey and Art are all out following up the clues as to the earth-boxes. I shall finish my round of work and we shall meet to-night.
\end{diary}

\section{Mina Harker's Journal}

\begin{diary}{1 October.}
It is strange to me to be kept in the dark as I am to-day; after Jonathan's full confidence for so many years, to see him manifestly avoid certain matters, and those the most vital of all. This morning I slept late after the fatigues of yesterday, and though Jonathan was late too, he was the earlier. He spoke to me before he went out, never more sweetly or tenderly, but he never mentioned a word of what had happened in the visit to the Count's house. And yet he must have known how terribly anxious I was. Poor dear fellow! I suppose it must have distressed him even more than it did me. They all agreed that it was best that I should not be drawn further into this awful work, and I acquiesced. But to think that he keeps anything from me! And now I am crying like a silly fool, when I \textit{know} it comes from my husband's great love and from the good, good wishes of those other strong men.

That has done me good. Well, some day Jonathan will tell me all; and lest it should ever be that he should think for a moment that I kept anything from him, I still keep my journal as usual. Then if he has feared of my trust I shall show it to him, with every thought of my heart put down for his dear eyes to read. I feel strangely sad and low-spirited to-day. I suppose it is the reaction from the terrible excitement.

Last night I went to bed when the men had gone, simply because they told me to. I didn't feel sleepy, and I did feel full of devouring anxiety. I kept thinking over everything that has been ever since Jonathan came to see me in London, and it all seems like a horrible tragedy, with fate pressing on relentlessly to some destined end. Everything that one does seems, no matter how right it may be, to bring on the very thing which is most to be deplored. If I hadn't gone to Whitby, perhaps poor dear Lucy would be with us now. She hadn't taken to visiting the churchyard till I came, and if she hadn't come there in the day-time with me she wouldn't have walked there in her sleep; and if she hadn't gone there at night and asleep, that monster couldn't have destroyed her as he did. Oh, why did I ever go to Whitby? There now, crying again! I wonder what has come over me to-day. I must hide it from Jonathan, for if he knew that I had been crying twice in one morning—I, who never cried on my own account, and whom he has never caused to shed a tear—the dear fellow would fret his heart out. I shall put a bold face on, and if I do feel weepy, he shall never see it. I suppose it is one of the lessons that we poor women have to learn\ellipsispunct{.}

I can't quite remember how I fell asleep last night. I remember hearing the sudden barking of the dogs and a lot of queer sounds, like praying on a very tumultuous scale, from Mr Renfield's room, which is somewhere under this. And then there was silence over everything, silence so profound that it startled me, and I got up and looked out of the window. All was dark and silent, the black shadows thrown by the moonlight seeming full of a silent mystery of their own. Not a thing seemed to be stirring, but all to be grim and fixed as death or fate; so that a thin streak of white mist, that crept with almost imperceptible slowness across the grass towards the house, seemed to have a sentience and a vitality of its own. I think that the digression of my thoughts must have done me good, for when I got back to bed I found a lethargy creeping over me. I lay a while, but could not quite sleep, so I got out and looked out of the window again. The mist was spreading, and was now close up to the house, so that I could see it lying thick against the wall, as though it were stealing up to the windows. The poor man was more loud than ever, and though I could not distinguish a word he said, I could in some way recognise in his tones some passionate entreaty on his part. Then there was the sound of a struggle, and I knew that the attendants were dealing with him. I was so frightened that I crept into bed, and pulled the clothes over my head, putting my fingers in my ears. I was not then a bit sleepy, at least so I thought; but I must have fallen asleep, for, except dreams, I do not remember anything until the morning, when Jonathan woke me. I think that it took me an effort and a little time to realise where I was, and that it was Jonathan who was bending over me. My dream was very peculiar, and was almost typical of the way that waking thoughts become merged in, or continued in, dreams.

I thought that I was asleep, and waiting for Jonathan to come back. I was very anxious about him, and I was powerless to act; my feet, and my hands, and my brain were weighted, so that nothing could proceed at the usual pace. And so I slept uneasily and thought. Then it began to dawn upon me that the air was heavy, and dank, and cold. I put back the clothes from my face, and found, to my surprise, that all was dim around. The gaslight which I had left lit for Jonathan, but turned down, came only like a tiny red spark through the fog, which had evidently grown thicker and poured into the room. Then it occurred to me that I had shut the window before I had come to bed. I would have got out to make certain on the point, but some leaden lethargy seemed to chain my limbs and even my will. I lay still and endured; that was all. I closed my eyes, but could still see through my eyelids. (It is wonderful what tricks our dreams play us, and how conveniently we can imagine.) The mist grew thicker and thicker and I could see now how it came in, for I could see it like smoke—or with the white energy of boiling water—pouring in, not through the window, but through the joinings of the door. It got thicker and thicker, till it seemed as if it became concentrated into a sort of pillar of cloud in the room, through the top of which I could see the light of the gas shining like a red eye. Things began to whirl through my brain just as the cloudy column was now whirling in the room, and through it all came the scriptural words <a pillar of cloud by day and of fire by night.> Was it indeed some such spiritual guidance that was coming to me in my sleep? But the pillar was composed of both the day and the night-guiding, for the fire was in the red eye, which at the thought got a new fascination for me; till, as I looked, the fire divided, and seemed to shine on me through the fog like two red eyes, such as Lucy told me of in her momentary mental wandering when, on the cliff, the dying sunlight struck the windows of St Mary's Church. Suddenly the horror burst upon me that it was thus that Jonathan had seen those awful women growing into reality through the whirling mist in the moonlight, and in my dream I must have fainted, for all became black darkness. The last conscious effort which imagination made was to show me a livid white face bending over me out of the mist. I must be careful of such dreams, for they would unseat one's reason if there were too much of them. I would get Dr Van Helsing or Dr Seward to prescribe something for me which would make me sleep, only that I fear to alarm them. Such a dream at the present time would become woven into their fears for me. To-night I shall strive hard to sleep naturally. If I do not, I shall to-morrow night get them to give me a dose of chloral; that cannot hurt me for once, and it will give me a good night's sleep. Last night tired me more than if I had not slept at all.
\end{diary}
 

\begin{diary}{2 October 10 \textsc{p.m.}}
Last night I slept, but did not dream. I must have slept soundly, for I was not waked by Jonathan coming to bed; but the sleep has not refreshed me, for to-day I feel terribly weak and spiritless. I spent all yesterday trying to read, or lying down dozing. In the afternoon Mr Renfield asked if he might see me. Poor man, he was very gentle, and when I came away he kissed my hand and bade God bless me. Some way it affected me much; I am crying when I think of him. This is a new weakness, of which I must be careful. Jonathan would be miserable if he knew I had been crying. He and the others were out till dinner-time, and they all came in tired. I did what I could to brighten them up, and I suppose that the effort did me good, for I forgot how tired I was. After dinner they sent me to bed, and all went off to smoke together, as they said, but I knew that they wanted to tell each other of what had occurred to each during the day; I could see from Jonathan's manner that he had something important to communicate. I was not so sleepy as I should have been; so before they went I asked Dr Seward to give me a little opiate of some kind, as I had not slept well the night before. He very kindly made me up a sleeping draught, which he gave to me, telling me that it would do me no harm, as it was very mild\ellipsispunct{.} I have taken it, and am waiting for sleep, which still keeps aloof. I hope I have not done wrong, for as sleep begins to flirt with me, a new fear comes: that I may have been foolish in thus depriving myself of the power of waking. I might want it. Here comes sleep. Good-night.
	\end{diary}