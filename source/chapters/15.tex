%!TeX root=../draculatop.tex
\chapter[Chapter \thechapter]{}

\section{Dr Seward's Diary—continued}

For a while sheer anger mastered me; it was as if he had during her life struck Lucy on the face. I smote the table hard and rose up as I said to him:—

<Dr Van Helsing, are you mad?> He raised his head and looked at me, and somehow the tenderness of his face calmed me at once. <Would I were!> he said. <Madness were easy to bear compared with truth like this. Oh, my friend, why, think you, did I go so far round, why take so long to tell you so simple a thing? Was it because I hate you and have hated you all my life? Was it because I wished to give you pain? Was it that I wanted, now so late, revenge for that time when you saved my life, and from a fearful death? Ah no!>

<Forgive me,> said I\@. He went on:—

<My friend, it was because I wished to be gentle in the breaking to you, for I know you have loved that so sweet lady. But even yet I do not expect you to believe. It is so hard to accept at once any abstract truth, that we may doubt such to be possible when we have always believed the <no> of it; it is more hard still to accept so sad a concrete truth, and of such a one as Miss Lucy. To-night I go to prove it. Dare you come with me?>

This staggered me. A man does not like to prove such a truth; Byron excepted from the category, jealousy.

<And prove the very truth he most abhorred.>
He saw my hesitation, and spoke:—

<The logic is simple, no madman's logic this time, jumping from tussock to tussock in a misty bog. If it be not true, then proof will be relief; at worst it will not harm. If it be true! Ah, there is the dread; yet very dread should help my cause, for in it is some need of belief. Come, I tell you what I propose: first, that we go off now and see that child in the hospital. Dr Vincent, of the North Hospital, where the papers say the child is, is friend of mine, and I think of yours since you were in class at Amsterdam. He will let two scientists see his case, if he will not let two friends. We shall tell him nothing, but only that we wish to learn. And then\longdash>

<And then?> He took a key from his pocket and held it up. <And then we spend the night, you and I, in the churchyard where Lucy lies. This is the key that lock the tomb. I had it from the coffin-man to give to Arthur.> My heart sank within me, for I felt that there was some fearful ordeal before us. I could do nothing, however, so I plucked up what heart I could and said that we had better hasten, as the afternoon was passing\ellipsispunct{.}

We found the child awake. It had had a sleep and taken some food, and altogether was going on well. Dr Vincent took the bandage from its throat, and showed us the punctures. There was no mistaking the similarity to those which had been on Lucy's throat. They were smaller, and the edges looked fresher; that was all. We asked Vincent to what he attributed them, and he replied that it must have been a bite of some animal, perhaps a rat; but, for his own part, he was inclined to think that it was one of the bats which are so numerous on the northern heights of London. <Out of so many harmless ones,> he said, <there may be some wild specimen from the South of a more malignant species. Some sailor may have brought one home, and it managed to escape; or even from the Zoölogical Gardens a young one may have got loose, or one be bred there from a vampire. These things do occur, you know. Only ten days ago a wolf got out, and was, I believe, traced up in this direction. For a week after, the children were playing nothing but Red Riding Hood on the Heath and in every alley in the place until this <bloofer lady> scare came along, since when it has been quite a gala-time with them. Even this poor little mite, when he woke up to-day, asked the nurse if he might go away. When she asked him why he wanted to go, he said he wanted to play with the <bloofer lady.>>

<I hope,> said Van Helsing, <that when you are sending the child home you will caution its parents to keep strict watch over it. These fancies to stray are most dangerous; and if the child were to remain out another night, it would probably be fatal. But in any case I suppose you will not let it away for some days?>

<Certainly not, not for a week at least; longer if the wound is not healed.>

Our visit to the hospital took more time than we had reckoned on, and the sun had dipped before we came out. When Van Helsing saw how dark it was, he said:—

<There is no hurry. It is more late than I thought. Come, let us seek somewhere that we may eat, and then we shall go on our way.>

We dined at <Jack Straw's Castle> along with a little crowd of bicyclists and others who were genially noisy. About ten o'clock we started from the inn. It was then very dark, and the scattered lamps made the darkness greater when we were once outside their individual radius. The Professor had evidently noted the road we were to go, for he went on unhesitatingly; but, as for me, I was in quite a mixup as to locality. As we went further, we met fewer and fewer people, till at last we were somewhat surprised when we met even the patrol of horse police going their usual suburban round. At last we reached the wall of the churchyard, which we climbed over. With some little difficulty—for it was very dark, and the whole place seemed so strange to us—we found the Westenra tomb. The Professor took the key, opened the creaky door, and standing back, politely, but quite unconsciously, motioned me to precede him. There was a delicious irony in the offer, in the courtliness of giving preference on such a ghastly occasion. My companion followed me quickly, and cautiously drew the door to, after carefully ascertaining that the lock was a falling, and not a spring, one. In the latter case we should have been in a bad plight. Then he fumbled in his bag, and taking out a matchbox and a piece of candle, proceeded to make a light. The tomb in the day-time, and when wreathed with fresh flowers, had looked grim and gruesome enough; but now, some days afterwards, when the flowers hung lank and dead, their whites turning to rust and their greens to browns; when the spider and the beetle had resumed their accustomed dominance; when time-discoloured stone, and dust-encrusted mortar, and rusty, dank iron, and tarnished brass, and clouded silver-plating gave back the feeble glimmer of a candle, the effect was more miserable and sordid than could have been imagined. It conveyed irresistibly the idea that life—animal life—was not the only thing which could pass away.

Van Helsing went about his work systematically. Holding his candle so that he could read the coffin plates, and so holding it that the sperm dropped in white patches which congealed as they touched the metal, he made assurance of Lucy's coffin. Another search in his bag, and he took out a turnscrew.

<What are you going to do?> I asked.

<To open the coffin. You shall yet be convinced.> Straightway he began taking out the screws, and finally lifted off the lid, showing the casing of lead beneath. The sight was almost too much for me. It seemed to be as much an affront to the dead as it would have been to have stripped off her clothing in her sleep whilst living; I actually took hold of his hand to stop him. He only said: <You shall see,> and again fumbling in his bag, took out a tiny fret-saw. Striking the turnscrew through the lead with a swift downward stab, which made me wince, he made a small hole, which was, however, big enough to admit the point of the saw. I had expected a rush of gas from the week-old corpse. We doctors, who have had to study our dangers, have to become accustomed to such things, and I drew back towards the door. But the Professor never stopped for a moment; he sawed down a couple of feet along one side of the lead coffin, and then across, and down the other side. Taking the edge of the loose flange, he bent it back towards the foot of the coffin, and holding up the candle into the aperture, motioned to me to look.

I drew near and looked. The coffin was empty.

It was certainly a surprise to me, and gave me a considerable shock, but Van Helsing was unmoved. He was now more sure than ever of his ground, and so emboldened to proceed in his task. <Are you satisfied now, friend John?> he asked.

I felt all the dogged argumentativeness of my nature awake within me as I answered him:—

<I am satisfied that Lucy's body is not in that coffin; but that only proves one thing.>

<And what is that, friend John?>

<That it is not there.>

<That is good logic,> he said, <so far as it goes. But how do you—how can you—account for it not being there?>

<Perhaps a body-snatcher,> I suggested. <Some of the undertaker's people may have stolen it.> I felt that I was speaking folly, and yet it was the only real cause which I could suggest. The Professor sighed. <Ah well!> he said, <we must have more proof. Come with me.>

He put on the coffin-lid again, gathered up all his things and placed them in the bag, blew out the light, and placed the candle also in the bag. We opened the door, and went out. Behind us he closed the door and locked it. He handed me the key, saying: <Will you keep it? You had better be assured.> I laughed—it was not a very cheerful laugh, I am bound to say—as I motioned him to keep it. <A key is nothing,> I said; <there may be duplicates; and anyhow it is not difficult to pick a lock of that kind.> He said nothing, but put the key in his pocket. Then he told me to watch at one side of the churchyard whilst he would watch at the other. I took up my place behind a yew-tree, and I saw his dark figure move until the intervening headstones and trees hid it from my sight.

It was a lonely vigil. Just after I had taken my place I heard a distant clock strike twelve, and in time came one and two. I was chilled and unnerved, and angry with the Professor for taking me on such an errand and with myself for coming. I was too cold and too sleepy to be keenly observant, and not sleepy enough to betray my trust so altogether I had a dreary, miserable time.

Suddenly, as I turned round, I thought I saw something like a white streak, moving between two dark yew-trees at the side of the churchyard farthest from the tomb; at the same time a dark mass moved from the Professor's side of the ground, and hurriedly went towards it. Then I too moved; but I had to go round headstones and railed-off tombs, and I stumbled over graves. The sky was overcast, and somewhere far off an early cock crew. A little way off, beyond a line of scattered juniper-trees, which marked the pathway to the church, a white, dim figure flitted in the direction of the tomb. The tomb itself was hidden by trees, and I could not see where the figure disappeared. I heard the rustle of actual movement where I had first seen the white figure, and coming over, found the Professor holding in his arms a tiny child. When he saw me he held it out to me, and said:—

<Are you satisfied now?>

<No,> I said, in a way that I felt was aggressive.

<Do you not see the child?>

<Yes, it is a child, but who brought it here? And is it wounded?> I asked.

<We shall see,> said the Professor, and with one impulse we took our way out of the churchyard, he carrying the sleeping child.

When we had got some little distance away, we went into a clump of trees, and struck a match, and looked at the child's throat. It was without a scratch or scar of any kind.

<Was I right?> I asked triumphantly.

<We were just in time,> said the Professor thankfully.

We had now to decide what we were to do with the child, and so consulted about it. If we were to take it to a police-station we should have to give some account of our movements during the night; at least, we should have had to make some statement as to how we had come to find the child. So finally we decided that we would take it to the Heath, and when we heard a policeman coming, would leave it where he could not fail to find it; we would then seek our way home as quickly as we could. All fell out well. At the edge of Hampstead Heath we heard a policeman's heavy tramp, and laying the child on the pathway, we waited and watched until he saw it as he flashed his lantern to and fro. We heard his exclamation of astonishment, and then we went away silently. By good chance we got a cab near the <Spaniards,> and drove to town.

I cannot sleep, so I make this entry. But I must try to get a few hours' sleep, as Van Helsing is to call for me at noon. He insists that I shall go with him on another expedition.

 

\begin{diary}{27 September.}
It was two o'clock before we found a suitable opportunity for our attempt. The funeral held at noon was all completed, and the last stragglers of the mourners had taken themselves lazily away, when, looking carefully from behind a clump of alder-trees, we saw the sexton lock the gate after him. We knew then that we were safe till morning did we desire it; but the Professor told me that we should not want more than an hour at most. Again I felt that horrid sense of the reality of things, in which any effort of imagination seemed out of place; and I realised distinctly the perils of the law which we were incurring in our unhallowed work. Besides, I felt it was all so useless. Outrageous as it was to open a leaden coffin, to see if a woman dead nearly a week were really dead, it now seemed the height of folly to open the tomb again, when we knew, from the evidence of our own eyesight, that the coffin was empty. I shrugged my shoulders, however, and rested silent, for Van Helsing had a way of going on his own road, no matter who remonstrated. He took the key, opened the vault, and again courteously motioned me to precede. The place was not so gruesome as last night, but oh, how unutterably mean-looking when the sunshine streamed in. Van Helsing walked over to Lucy's coffin, and I followed. He bent over and again forced back the leaden flange; and then a shock of surprise and dismay shot through me.

There lay Lucy, seemingly just as we had seen her the night before her funeral. She was, if possible, more radiantly beautiful than ever; and I could not believe that she was dead. The lips were red, nay redder than before; and on the cheeks was a delicate bloom.

<Is this a juggle?> I said to him.

<Are you convinced now?> said the Professor in response, and as he spoke he put over his hand, and in a way that made me shudder, pulled back the dead lips and showed the white teeth.

<See,> he went on, <see, they are even sharper than before. With this and this>—and he touched one of the canine teeth and that below it—<the little children can be bitten. Are you of belief now, friend John?> Once more, argumentative hostility woke within me. I \textit{could} not accept such an overwhelming idea as he suggested; so, with an attempt to argue of which I was even at the moment ashamed, I said:—

<She may have been placed here since last night.>

<Indeed? That is so, and by whom?>

<I do not know. Some one has done it.>

<And yet she has been dead one week. Most peoples in that time would not look so.> I had no answer for this, so was silent. Van Helsing did not seem to notice my silence; at any rate, he showed neither chagrin nor triumph. He was looking intently at the face of the dead woman, raising the eyelids and looking at the eyes, and once more opening the lips and examining the teeth. Then he turned to me and said:—

<Here, there is one thing which is different from all recorded; here is some dual life that is not as the common. She was bitten by the vampire when she was in a trance, sleep-walking—oh, you start; you do not know that, friend John, but you shall know it all later—and in trance could he best come to take more blood. In trance she died, and in trance she is Un-Dead, too. So it is that she differ from all other. Usually when the Un-Dead sleep at home>—as he spoke he made a comprehensive sweep of his arm to designate what to a vampire was <home>—<their face show what they are, but this so sweet that was when she not Un-Dead she go back to the nothings of the common dead. There is no malign there, see, and so it make hard that I must kill her in her sleep.> This turned my blood cold, and it began to dawn upon me that I was accepting Van Helsing's theories; but if she were really dead, what was there of terror in the idea of killing her? He looked up at me, and evidently saw the change in my face, for he said almost joyously:—

<Ah, you believe now?>

I answered: <Do not press me too hard all at once. I am willing to accept. How will you do this bloody work?>

<I shall cut off her head and fill her mouth with garlic, and I shall drive a stake through her body.> It made me shudder to think of so mutilating the body of the woman whom I had loved. And yet the feeling was not so strong as I had expected. I was, in fact, beginning to shudder at the presence of this being, this Un-Dead, as Van Helsing called it, and to loathe it. Is it possible that love is all subjective, or all objective?

I waited a considerable time for Van Helsing to begin, but he stood as if wrapped in thought. Presently he closed the catch of his bag with a snap, and said:—

<I have been thinking, and have made up my mind as to what is best. If I did simply follow my inclining I would do now, at this moment, what is to be done; but there are other things to follow, and things that are thousand times more difficult in that them we do not know. This is simple. She have yet no life taken, though that is of time; and to act now would be to take danger from her for ever. But then we may have to want Arthur, and how shall we tell him of this? If you, who saw the wounds on Lucy's throat, and saw the wounds so similar on the child's at the hospital; if you, who saw the coffin empty last night and full to-day with a woman who have not change only to be more rose and more beautiful in a whole week, after she die—if you know of this and know of the white figure last night that brought the child to the churchyard, and yet of your own senses you did not believe, how, then, can I expect Arthur, who know none of those things, to believe? He doubted me when I took him from her kiss when she was dying. I know he has forgiven me because in some mistaken idea I have done things that prevent him say good-bye as he ought; and he may think that in some more mistaken idea this woman was buried alive; and that in most mistake of all we have killed her. He will then argue back that it is we, mistaken ones, that have killed her by our ideas; and so he will be much unhappy always. Yet he never can be sure; and that is the worst of all. And he will sometimes think that she he loved was buried alive, and that will paint his dreams with horrors of what she must have suffered; and again, he will think that we may be right, and that his so beloved was, after all, an Un-Dead. No! I told him once, and since then I learn much. Now, since I know it is all true, a hundred thousand times more do I know that he must pass through the bitter waters to reach the sweet. He, poor fellow, must have one hour that will make the very face of heaven grow black to him; then we can act for good all round and send him peace. My mind is made up. Let us go. You return home for to-night to your asylum, and see that all be well. As for me, I shall spend the night here in this churchyard in my own way. To-morrow night you will come to me to the Berkeley Hotel at ten of the clock. I shall send for Arthur to come too, and also that so fine young man of America that gave his blood. Later we shall all have work to do. I come with you so far as Piccadilly and there dine, for I must be back here before the sun set.>

So we locked the tomb and came away, and got over the wall of the churchyard, which was not much of a task, and drove back to Piccadilly.

\end{diary}

\section{Note left by Van Helsing in his portmanteau, Berkeley Hotel directed to John Seward, M\@.~D\@.}
	\begin{center}\itshape (Not delivered.)\end{center}
	
	\begin{mail}{27 September.}{Friend John,}

I write this in case anything should happen. I go alone to watch in that churchyard. It pleases me that the Un-Dead, Miss Lucy, shall not leave to-night, that so on the morrow night she may be more eager. Therefore I shall fix some things she like not—garlic and a crucifix—and so seal up the door of the tomb. She is young as Un-Dead, and will heed. Moreover, these are only to prevent her coming out; they may not prevail on her wanting to get in; for then the Un-Dead is desperate, and must find the line of least resistance, whatsoever it may be. I shall be at hand all the night from sunset till after the sunrise, and if there be aught that may be learned I shall learn it. For Miss Lucy or from her, I have no fear; but that other to whom is there that she is Un-Dead, he have now the power to seek her tomb and find shelter. He is cunning, as I know from Mr Jonathan and from the way that all along he have fooled us when he played with us for Miss Lucy's life, and we lost; and in many ways the Un-Dead are strong. He have always the strength in his hand of twenty men; even we four who gave our strength to Miss Lucy it also is all to him. Besides, he can summon his wolf and I know not what. So if it be that he come thither on this night he shall find me; but none other shall—until it be too late. But it may be that he will not attempt the place. There is no reason why he should; his hunting ground is more full of game than the churchyard where the Un-Dead woman sleep, and the one old man watch.

Therefore I write this in case\ellipsispunct{.} Take the papers that are with this, the diaries of Harker and the rest, and read them, and then find this great Un-Dead, and cut off his head and burn his heart or drive a stake through it, so that the world may rest from him.

\closeletter[If it be so, farewell.]{Van Helsing.}
\end{mail}

\section{Dr Seward's Diary}

\begin{diary}{28 September.}
It is wonderful what a good night's sleep will do for one. Yesterday I was almost willing to accept Van Helsing's monstrous ideas; but now they seem to start out lurid before me as outrages on common sense. I have no doubt that he believes it all. I wonder if his mind can have become in any way unhinged. Surely there must be \textit{some} rational explanation of all these mysterious things. Is it possible that the Professor can have done it himself? He is so abnormally clever that if he went off his head he would carry out his intent with regard to some fixed idea in a wonderful way. I am loath to think it, and indeed it would be almost as great a marvel as the other to find that Van Helsing was mad; but anyhow I shall watch him carefully. I may get some light on the mystery.
\end{diary}
 

\begin{diary}{29 September, morning}
Last night, at a little before ten o'clock, Arthur and Quincey came into Van Helsing's room; he told us all that he wanted us to do, but especially addressing himself to Arthur, as if all our wills were centred in his. He began by saying that he hoped we would all come with him too, <for,> he said, <there is a grave duty to be done there. You were doubtless surprised at my letter?> This query was directly addressed to Lord Godalming.

<I was. It rather upset me for a bit. There has been so much trouble around my house of late that I could do without any more. I have been curious, too, as to what you mean. Quincey and I talked it over; but the more we talked, the more puzzled we got, till now I can say for myself that I'm about up a tree as to any meaning about anything.>

<Me too,> said Quincey Morris laconically.

<Oh,> said the Professor, <then you are nearer the beginning, both of you, than friend John here, who has to go a long way back before he can even get so far as to begin.>

It was evident that he recognised my return to my old doubting frame of mind without my saying a word. Then, turning to the other two, he said with intense gravity:—

<I want your permission to do what I think good this night. It is, I know, much to ask; and when you know what it is I propose to do you will know, and only then, how much. Therefore may I ask that you promise me in the dark, so that afterwards, though you may be angry with me for a time—I must not disguise from myself the possibility that such may be—you shall not blame yourselves for anything.>

<That's frank anyhow,> broke in Quincey. <I'll answer for the Professor. I don't quite see his drift, but I swear he's honest; and that's good enough for me.>

<I thank you, sir,> said Van Helsing proudly. <I have done myself the honour of counting you one trusting friend, and such endorsement is dear to me.> He held out a hand, which Quincey took.

Then Arthur spoke out:—

<Dr Van Helsing, I don't quite like to <buy a pig in a poke,> as they say in Scotland, and if it be anything in which my honour as a gentleman or my faith as a Christian is concerned, I cannot make such a promise. If you can assure me that what you intend does not violate either of these two, then I give my consent at once; though for the life of me, I cannot understand what you are driving at.>

<I accept your limitation,> said Van Helsing, <and all I ask of you is that if you feel it necessary to condemn any act of mine, you will first consider it well and be satisfied that it does not violate your reservations.>

<Agreed!> said Arthur; <that is only fair. And now that the \textit{pourparlers} are over, may I ask what it is we are to do?>

<I want you to come with me, and to come in secret, to the churchyard at Kingstead.>

Arthur's face fell as he said in an amazed sort of way:—

<Where poor Lucy is buried?> The Professor bowed. Arthur went on: <And when there?>

<To enter the tomb!> Arthur stood up.

<Professor, are you in earnest; or it is some monstrous joke? Pardon me, I see that you are in earnest.> He sat down again, but I could see that he sat firmly and proudly, as one who is on his dignity. There was silence until he asked again:—

<And when in the tomb?>

<To open the coffin.>

<This is too much!> he said, angrily rising again. <I am willing to be patient in all things that are reasonable; but in this—this desecration of the grave—of one who\longdash> He fairly choked with indignation. The Professor looked pityingly at him.

<If I could spare you one pang, my poor friend,> he said, <God knows I would. But this night our feet must tread in thorny paths; or later, and for ever, the feet you love must walk in paths of flame!>

Arthur looked up with set white face and said:—

<Take care, sir, take care!>

<Would it not be well to hear what I have to say?> said Van Helsing. <And then you will at least know the limit of my purpose. Shall I go on?>

<That's fair enough,> broke in Morris.

After a pause Van Helsing went on, evidently with an effort:—

<Miss Lucy is dead; is it not so? Yes! Then there can be no wrong to her. But if she be not dead\longdash>

Arthur jumped to his feet.

<Good God!> he cried. <What do you mean? Has there been any mistake; has she been buried alive?> He groaned in anguish that not even hope could soften.

<I did not say she was alive, my child; I did not think it. I go no further than to say that she might be Un-Dead.>

<Un-Dead! Not alive! What do you mean? Is this all a nightmare, or what is it?>

<There are mysteries which men can only guess at, which age by age they may solve only in part. Believe me, we are now on the verge of one. But I have not done. May I cut off the head of dead Miss Lucy?>

<Heavens and earth, no!> cried Arthur in a storm of passion. <Not for the wide world will I consent to any mutilation of her dead body. Dr Van Helsing, you try me too far. What have I done to you that you should torture me so? What did that poor, sweet girl do that you should want to cast such dishonour on her grave? Are you mad to speak such things, or am I mad to listen to them? Don't dare to think more of such a desecration; I shall not give my consent to anything you do. I have a duty to do in protecting her grave from outrage; and, by God, I shall do it!>

Van Helsing rose up from where he had all the time been seated, and said, gravely and sternly:—

<My Lord Godalming, I, too, have a duty to do, a duty to others, a duty to you, a duty to the dead; and, by God, I shall do it! All I ask you now is that you come with me, that you look and listen; and if when later I make the same request you do not be more eager for its fulfilment even than I am, then—then I shall do my duty, whatever it may seem to me. And then, to follow of your Lordship's wishes I shall hold myself at your disposal to render an account to you, when and where you will.> His voice broke a little, and he went on with a voice full of pity:—

<But, I beseech you, do not go forth in anger with me. In a long life of acts which were often not pleasant to do, and which sometimes did wring my heart, I have never had so heavy a task as now. Believe me that if the time comes for you to change your mind towards me, one look from you will wipe away all this so sad hour, for I would do what a man can to save you from sorrow. Just think. For why should I give myself so much of labour and so much of sorrow? I have come here from my own land to do what I can of good; at the first to please my friend John, and then to help a sweet young lady, whom, too, I came to love. For her—I am ashamed to say so much, but I say it in kindness—I gave what you gave; the blood of my veins; I gave it, I, who was not, like you, her lover, but only her physician and her friend. I gave to her my nights and days—before death, after death; and if my death can do her good even now, when she is the dead Un-Dead, she shall have it freely.> He said this with a very grave, sweet pride, and Arthur was much affected by it. He took the old man's hand and said in a broken voice:—

<Oh, it is hard to think of it, and I cannot understand; but at least I shall go with you and wait.>
\end{diary}