%!TeX root=../draculatop.tex
\chapter[Chapter \thechapter]{}


\section{Dr Seward's Phonograph Diary, Spoken By Van Helsing}

This to Jonathan Harker.

You are to stay with your dear Madam Mina. We shall go to make our search—if I can call it so, for it is not search but knowing, and we seek confirmation only. But do you stay and take care of her to-day. This is your best and most holiest office. This day nothing can find him here. Let me tell you that so you will know what we four know already, for I have tell them. He, our enemy, have gone away; he have gone back to his Castle in Transylvania. I know it so well, as if a great hand of fire wrote it on the wall. He have prepare for this in some way, and that last earth-box was ready to ship somewheres. For this he took the money; for this he hurry at the last, lest we catch him before the sun go down. It was his last hope, save that he might hide in the tomb that he think poor Miss Lucy, being as he thought like him, keep open to him. But there was not of time. When that fail he make straight for his last resource—his last earth-work I might say did I wish \textit{double entente}. He is clever, oh, so clever! he know that his game here was finish; and so he decide he go back home. He find ship going by the route he came, and he go in it. We go off now to find what ship, and whither bound; when we have discover that, we come back and tell you all. Then we will comfort you and poor dear Madam Mina with new hope. For it will be hope when you think it over: that all is not lost. This very creature that we pursue, he take hundreds of years to get so far as London; and yet in one day, when we know of the disposal of him we drive him out. He is finite, though he is powerful to do much harm and suffers not as we do. But we are strong, each in our purpose; and we are all more strong together. Take heart afresh, dear husband of Madam Mina. This battle is but begun, and in the end we shall win—so sure as that God sits on high to watch over His children. Therefore be of much comfort till we return.

\begin{flushright}Van Helsing.\end{flushright}

\section{Jonathan Harker's Journal}

\begin{diary}{4 October.}
When I read to Mina, Van Helsing's message in the phonograph, the poor girl brightened up considerably. Already the certainty that the Count is out of the country has given her comfort; and comfort is strength to her. For my own part, now that his horrible danger is not face to face with us, it seems almost impossible to believe in it. Even my own terrible experiences in Castle Dracula seem like a long-forgotten dream. Here in the crisp autumn air in the bright sunlight—

Alas! how can I disbelieve! In the midst of my thought my eye fell on the red scar on my poor darling's white forehead. Whilst that lasts, there can be no disbelief. And afterwards the very memory of it will keep faith crystal clear. Mina and I fear to be idle, so we have been over all the diaries again and again. Somehow, although the reality seems greater each time, the pain and the fear seem less. There is something of a guiding purpose manifest throughout, which is comforting. Mina says that perhaps we are the instruments of ultimate good. It may be! I shall try to think as she does. We have never spoken to each other yet of the future. It is better to wait till we see the Professor and the others after their investigations.

The day is running by more quickly than I ever thought a day could run for me again. It is now three o'clock.
\end{diary}

\section{Mina Harker's Journal}

\begin{diary}{5 October, 5 \textsc{p.m.}}
Our meeting for report. Present: Professor Van Helsing, Lord Godalming, Dr Seward, Mr Quincey Morris, Jonathan Harker, Mina Harker.

Dr Van Helsing described what steps were taken during the day to discover on what boat and whither bound Count Dracula made his escape:—

<As I knew that he wanted to get back to Transylvania, I felt sure that he must go by the Danube mouth; or by somewhere in the Black Sea, since by that way he come. It was a dreary blank that was before us. \textit{Omne ignotum pro magnifico}; and so with heavy hearts we start to find what ships leave for the Black Sea last night. He was in sailing ship, since Madam Mina tell of sails being set. These not so important as to go in your list of the shipping in the \textit{Times}, and so we go, by suggestion of Lord Godalming, to your Lloyd's, where are note of all ships that sail, however so small. There we find that only one Black-Sea-bound ship go out with the tide. She is the \textit{Czarina Catherine}, and she sail from Doolittle's Wharf for Varna, and thence on to other parts and up the Danube. <Soh!> said I, <this is the ship whereon is the Count.> So off we go to Doolittle's Wharf, and there we find a man in an office of wood so small that the man look bigger than the office. From him we inquire of the goings of the \textit{Czarina Catherine}. He swear much, and he red face and loud of voice, but he good fellow all the same; and when Quincey give him something from his pocket which crackle as he roll it up, and put it in a so small bag which he have hid deep in his clothing, he still better fellow and humble servant to us. He come with us, and ask many men who are rough and hot; these be better fellows too when they have been no more thirsty. They say much of blood and bloom, and of others which I comprehend not, though I guess what they mean; but nevertheless they tell us all things which we want to know.

They make known to us among them, how last afternoon at about five o'clock comes a man so hurry. A tall man, thin and pale, with high nose and teeth so white, and eyes that seem to be burning. That he be all in black, except that he have a hat of straw which suit not him or the time. That he scatter his money in making quick inquiry as to what ship sails for the Black Sea and for where. Some took him to the office and then to the ship, where he will not go aboard but halt at shore end of gang-plank, and ask that the captain come to him. The captain come, when told that he will be pay well; and though he swear much at the first he agree to term. Then the thin man go and some one tell him where horse and cart can be hired. He go there and soon he come again, himself driving cart on which a great box; this he himself lift down, though it take several to put it on truck for the ship. He give much talk to captain as to how and where his box is to be place; but the captain like it not and swear at him in many tongues, and tell him that if he like he can come and see where it shall be. But he say <no>; that he come not yet, for that he have much to do. Whereupon the captain tell him that he had better be quick—with blood—for that his ship will leave the place—of blood—before the turn of the tide—with blood. Then the thin man smile and say that of course he must go when he think fit; but he will be surprise if he go quite so soon. The captain swear again, polyglot, and the thin man make him bow, and thank him, and say that he will so far intrude on his kindness as to come aboard before the sailing. Final the captain, more red than ever, and in more tongues tell him that he doesn't want no Frenchmen—with bloom upon them and also with blood—in his ship—with blood on her also. And so, after asking where there might be close at hand a ship where he might purchase ship forms, he departed.

No one knew where he went <or bloomin' well cared,> as they said, for they had something else to think of—well with blood again; for it soon became apparent to all that the \textit{Czarina Catherine} would not sail as was expected. A thin mist began to creep up from the river, and it grew, and grew; till soon a dense fog enveloped the ship and all around her. The captain swore polyglot—very polyglot—polyglot with bloom and blood; but he could do nothing. The water rose and rose; and he began to fear that he would lose the tide altogether. He was in no friendly mood, when just at full tide, the thin man came up the gang-plank again and asked to see where his box had been stowed. Then the captain replied that he wished that he and his box—old and with much bloom and blood—were in hell. But the thin man did not be offend, and went down with the mate and saw where it was place, and came up and stood awhile on deck in fog. He must have come off by himself, for none notice him. Indeed they thought not of him; for soon the fog begin to melt away, and all was clear again. My friends of the thirst and the language that was of bloom and blood laughed, as they told how the captain's swears exceeded even his usual polyglot, and was more than ever full of picturesque, when on questioning other mariners who were on movement up and down on the river that hour, he found that few of them had seen any of fog at all, except where it lay round the wharf. However, the ship went out on the ebb tide; and was doubtless by morning far down the river mouth. She was by then, when they told us, well out to sea.

And so, my dear Madam Mina, it is that we have to rest for a time, for our enemy is on the sea, with the fog at his command, on his way to the Danube mouth. To sail a ship takes time, go she never so quick; and when we start we go on land more quick, and we meet him there. Our best hope is to come on him when in the box between sunrise and sunset; for then he can make no struggle, and we may deal with him as we should. There are days for us, in which we can make ready our plan. We know all about where he go; for we have seen the owner of the ship, who have shown us invoices and all papers that can be. The box we seek is to be landed in Varna, and to be given to an agent, one Ristics who will there present his credentials; and so our merchant friend will have done his part. When he ask if there be any wrong, for that so, he can telegraph and have inquiry made at Varna, we say <no>; for what is to be done is not for police or of the customs. It must be done by us alone and in our own way.>

When Dr Van Helsing had done speaking, I asked him if he were certain that the Count had remained on board the ship. He replied: <We have the best proof of that: your own evidence, when in the hypnotic trance this morning.> I asked him again if it were really necessary that they should pursue the Count, for oh! I dread Jonathan leaving me, and I know that he would surely go if the others went. He answered in growing passion, at first quietly. As he went on, however, he grew more angry and more forceful, till in the end we could not but see wherein was at least some of that personal dominance which made him so long a master amongst men:—

<Yes, it is necessary—necessary—necessary! For your sake in the first, and then for the sake of humanity. This monster has done much harm already, in the narrow scope where he find himself, and in the short time when as yet he was only as a body groping his so small measure in darkness and not knowing. All this have I told these others; you, my dear Madam Mina, will learn it in the phonograph of my friend John, or in that of your husband. I have told them how the measure of leaving his own barren land—barren of peoples—and coming to a new land where life of man teems till they are like the multitude of standing corn, was the work of centuries. Were another of the Un-Dead, like him, to try to do what he has done, perhaps not all the centuries of the world that have been, or that will be, could aid him. With this one, all the forces of nature that are occult and deep and strong must have worked together in some wondrous way. The very place, where he have been alive, Un-Dead for all these centuries, is full of strangeness of the geologic and chemical world. There are deep caverns and fissures that reach none know whither. There have been volcanoes, some of whose openings still send out waters of strange properties, and gases that kill or make to vivify. Doubtless, there is something magnetic or electric in some of these combinations of occult forces which work for physical life in strange way; and in himself were from the first some great qualities. In a hard and warlike time he was celebrate that he have more iron nerve, more subtle brain, more braver heart, than any man. In him some vital principle have in strange way found their utmost; and as his body keep strong and grow and thrive, so his brain grow too. All this without that diabolic aid which is surely to him; for it have to yield to the powers that come from, and are, symbolic of good. And now this is what he is to us. He have infect you—oh, forgive me, my dear, that I must say such; but it is for good of you that I speak. He infect you in such wise, that even if he do no more, you have only to live—to live in your own old, sweet way; and so in time, death, which is of man's common lot and with God's sanction, shall make you like to him. This must not be! We have sworn together that it must not. Thus are we ministers of God's own wish: that the world, and men for whom His Son die, will not be given over to monsters, whose very existence would defame Him. He have allowed us to redeem one soul already, and we go out as the old knights of the Cross to redeem more. Like them we shall travel towards the sunrise; and like them, if we fall, we fall in good cause.> He paused and I said:—

<But will not the Count take his rebuff wisely? Since he has been driven from England, will he not avoid it, as a tiger does the village from which he has been hunted?>

<Aha!> he said, <your simile of the tiger good, for me, and I shall adopt him. Your man-eater, as they of India call the tiger who has once tasted blood of the human, care no more for the other prey, but prowl unceasing till he get him. This that we hunt from our village is a tiger, too, a man-eater, and he never cease to prowl. Nay, in himself he is not one to retire and stay afar. In his life, his living life, he go over the Turkey frontier and attack his enemy on his own ground; he be beaten back, but did he stay? No! He come again, and again, and again. Look at his persistence and endurance. With the child-brain that was to him he have long since conceive the idea of coming to a great city. What does he do? He find out the place of all the world most of promise for him. Then he deliberately set himself down to prepare for the task. He find in patience just how is his strength, and what are his powers. He study new tongues. He learn new social life; new environment of old ways, the politic, the law, the finance, the science, the habit of a new land and a new people who have come to be since he was. His glimpse that he have had, whet his appetite only and enkeen his desire. Nay, it help him to grow as to his brain; for it all prove to him how right he was at the first in his surmises. He have done this alone; all alone! from a ruin tomb in a forgotten land. What more may he not do when the greater world of thought is open to him. He that can smile at death, as we know him; who can flourish in the midst of diseases that kill off whole peoples. Oh, if such an one was to come from God, and not the Devil, what a force for good might he not be in this old world of ours. But we are pledged to set the world free. Our toil must be in silence, and our efforts all in secret; for in this enlightened age, when men believe not even what they see, the doubting of wise men would be his greatest strength. It would be at once his sheath and his armour, and his weapons to destroy us, his enemies, who are willing to peril even our own souls for the safety of one we love—for the good of mankind, and for the honour and glory of God.>

After a general discussion it was determined that for to-night nothing be definitely settled; that we should all sleep on the facts, and try to think out the proper conclusions. To-morrow, at breakfast, we are to meet again, and, after making our conclusions known to one another, we shall decide on some definite cause of action.

\divider

I feel a wonderful peace and rest to-night. It is as if some haunting presence were removed from me. Perhaps \textellipsis

My surmise was not finished, could not be; for I caught sight in the mirror of the red mark upon my forehead; and I knew that I was still unclean.
\end{diary}

\section{Dr Seward's Diary}

\begin{diary}{5 October.}
We all rose early, and I think that sleep did much for each and all of us. When we met at early breakfast there was more general cheerfulness than any of us had ever expected to experience again.

It is really wonderful how much resilience there is in human nature. Let any obstructing cause, no matter what, be removed in any way—even by death—and we fly back to first principles of hope and enjoyment. More than once as we sat around the table, my eyes opened in wonder whether the whole of the past days had not been a dream. It was only when I caught sight of the red blotch on Mrs Harker's forehead that I was brought back to reality. Even now, when I am gravely revolving the matter, it is almost impossible to realise that the cause of all our trouble is still existent. Even Mrs Harker seems to lose sight of her trouble for whole spells; it is only now and again, when something recalls it to her mind, that she thinks of her terrible scar. We are to meet here in my study in half an hour and decide on our course of action. I see only one immediate difficulty, I know it by instinct rather than reason: we shall all have to speak frankly; and yet I fear that in some mysterious way poor Mrs Harker's tongue is tied. I \textit{know} that she forms conclusions of her own, and from all that has been I can guess how brilliant and how true they must be; but she will not, or cannot, give them utterance. I have mentioned this to Van Helsing, and he and I are to talk it over when we are alone. I suppose it is some of that horrid poison which has got into her veins beginning to work. The Count had his own purposes when he gave her what Van Helsing called <the Vampire's baptism of blood.> Well, there may be a poison that distils itself out of good things; in an age when the existence of ptomaines is a mystery we should not wonder at anything! One thing I know: that if my instinct be true regarding poor Mrs Harker's silences, then there is a terrible difficulty—an unknown danger—in the work before us. The same power that compels her silence may compel her speech. I dare not think further; for so I should in my thoughts dishonour a noble woman!

Van Helsing is coming to my study a little before the others. I shall try to open the subject with him.
\end{diary}
 

\begin{diary}{Later.}
When the Professor came in, we talked over the state of things. I could see that he had something on his mind which he wanted to say, but felt some hesitancy about broaching the subject. After beating about the bush a little, he said suddenly:—

<Friend John, there is something that you and I must talk of alone, just at the first at any rate. Later, we may have to take the others into our confidence>; then he stopped, so I waited; he went on:—

<Madam Mina, our poor, dear Madam Mina is changing.> A cold shiver ran through me to find my worst fears thus endorsed. Van Helsing continued:—

<With the sad experience of Miss Lucy, we must this time be warned before things go too far. Our task is now in reality more difficult than ever, and this new trouble makes every hour of the direst importance. I can see the characteristics of the vampire coming in her face. It is now but very, very slight; but it is to be seen if we have eyes to notice without to prejudge. Her teeth are some sharper, and at times her eyes are more hard. But these are not all, there is to her the silence now often; as so it was with Miss Lucy. She did not speak, even when she wrote that which she wished to be known later. Now my fear is this. If it be that she can, by our hypnotic trance, tell what the Count see and hear, is it not more true that he who have hypnotise her first, and who have drink of her very blood and make her drink of his, should, if he will, compel her mind to disclose to him that which she know?> I nodded acquiescence; he went on:—

<Then, what we must do is to prevent this; we must keep her ignorant of our intent, and so she cannot tell what she know not. This is a painful task! Oh, so painful that it heart-break me to think of; but it must be. When to-day we meet, I must tell her that for reason which we will not to speak she must not more be of our council, but be simply guarded by us.> He wiped his forehead, which had broken out in profuse perspiration at the thought of the pain which he might have to inflict upon the poor soul already so tortured. I knew that it would be some sort of comfort to him if I told him that I also had come to the same conclusion; for at any rate it would take away the pain of doubt. I told him, and the effect was as I expected.

It is now close to the time of our general gathering. Van Helsing has gone away to prepare for the meeting, and his painful part of it. I really believe his purpose is to be able to pray alone.
\end{diary}
 

\begin{diary}{Later.}
At the very outset of our meeting a great personal relief was experienced by both Van Helsing and myself. Mrs Harker had sent a message by her husband to say that she would not join us at present, as she thought it better that we should be free to discuss our movements without her presence to embarrass us. The Professor and I looked at each other for an instant, and somehow we both seemed relieved. For my own part, I thought that if Mrs Harker realised the danger herself, it was much pain as well as much danger averted. Under the circumstances we agreed, by a questioning look and answer, with finger on lip, to preserve silence in our suspicions, until we should have been able to confer alone again. We went at once into our Plan of Campaign. Van Helsing roughly put the facts before us first:—

<The \textit{Czarina Catherine} left the Thames yesterday morning. It will take her at the quickest speed she has ever made at least three weeks to reach Varna; but we can travel overland to the same place in three days. Now, if we allow for two days less for the ship's voyage, owing to such weather influences as we know that the Count can bring to bear; and if we allow a whole day and night for any delays which may occur to us, then we have a margin of nearly two weeks. Thus, in order to be quite safe, we must leave here on 17th at latest. Then we shall at any rate be in Varna a day before the ship arrives, and able to make such preparations as may be necessary. Of course we shall all go armed—armed against evil things, spiritual as well as physical.> Here Quincey Morris added:—

<I understand that the Count comes from a wolf country, and it may be that he shall get there before us. I propose that we add Winchesters to our armament. I have a kind of belief in a Winchester when there is any trouble of that sort around. Do you remember, Art, when we had the pack after us at Tobolsk? What wouldn't we have given then for a repeater apiece!>

<Good!> said Van Helsing, <Winchesters it shall be. Quincey's head is level at all times, but most so when there is to hunt, metaphor be more dishonour to science than wolves be of danger to man. In the meantime we can do nothing here; and as I think that Varna is not familiar to any of us, why not go there more soon? It is as long to wait here as there. To-night and to-morrow we can get ready, and then, if all be well, we four can set out on our journey.>

<We four?> said Harker interrogatively, looking from one to another of us.

<Of course!> answered the Professor quickly, <you must remain to take care of your so sweet wife!> Harker was silent for awhile and then said in a hollow voice:—

<Let us talk of that part of it in the morning. I want to consult with Mina.> I thought that now was the time for Van Helsing to warn him not to disclose our plans to her; but he took no notice. I looked at him significantly and coughed. For answer he put his finger on his lips and turned away.
\end{diary}

\section{Jonathan Harker's Journal}

\begin{diary}{5 October, afternoon.}
For some time after our meeting this morning I could not think. The new phases of things leave my mind in a state of wonder which allows no room for active thought. Mina's determination not to take any part in the discussion set me thinking; and as I could not argue the matter with her, I could only guess. I am as far as ever from a solution now. The way the others received it, too, puzzled me; the last time we talked of the subject we agreed that there was to be no more concealment of anything amongst us. Mina is sleeping now, calmly and sweetly like a little child. Her lips are curved and her face beams with happiness. Thank God, there are such moments still for her.
\end{diary}
 

\begin{diary}{Later.}
How strange it all is. I sat watching Mina's happy sleep, and came as near to being happy myself as I suppose I shall ever be. As the evening drew on, and the earth took its shadows from the sun sinking lower, the silence of the room grew more and more solemn to me. All at once Mina opened her eyes, and looking at me tenderly, said:—

<Jonathan, I want you to promise me something on your word of honour. A promise made to me, but made holily in God's hearing, and not to be broken though I should go down on my knees and implore you with bitter tears. Quick, you must make it to me at once.>

<Mina,> I said, <a promise like that, I cannot make at once. I may have no right to make it.>

<But, dear one,> she said, with such spiritual intensity that her eyes were like pole stars, <it is I who wish it; and it is not for myself. You can ask Dr Van Helsing if I am not right; if he disagrees you may do as you will. Nay, more, if you all agree, later, you are absolved from the promise.>

<I promise!> I said, and for a moment she looked supremely happy; though to me all happiness for her was denied by the red scar on her forehead. She said:—

<Promise me that you will not tell me anything of the plans formed for the campaign against the Count. Not by word, or inference, or implication; not at any time whilst this remains to me!> and she solemnly pointed to the scar. I saw that she was in earnest, and said solemnly:—

<I promise!> and as I said it I felt that from that instant a door had been shut between us.
\end{diary}
 

\begin{diary}{Later, midnight.}
Mina has been bright and cheerful all the evening. So much so that all the rest seemed to take courage, as if infected somewhat with her gaiety; as a result even I myself felt as if the pall of gloom which weighs us down were somewhat lifted. We all retired early. Mina is now sleeping like a little child; it is a wonderful thing that her faculty of sleep remains to her in the midst of her terrible trouble. Thank God for it, for then at least she can forget her care. Perhaps her example may affect me as her gaiety did to-night. I shall try it. Oh! for a dreamless sleep.
\end{diary}
 

\begin{diary}{6 October, morning.}
Another surprise. Mina woke me early, about the same time as yesterday, and asked me to bring Dr Van Helsing. I thought that it was another occasion for hypnotism, and without question went for the Professor. He had evidently expected some such call, for I found him dressed in his room. His door was ajar, so that he could hear the opening of the door of our room. He came at once; as he passed into the room, he asked Mina if the others might come, too.

<No,> she said quite simply, <it will not be necessary. You can tell them just as well. I must go with you on your journey.>

Dr Van Helsing was as startled as I was. After a moment's pause he asked:—

<But why?>

<You must take me with you. I am safer with you, and you shall be safer, too.>

<But why, dear Madam Mina? You know that your safety is our solemnest duty. We go into danger, to which you are, or may be, more liable than any of us from—from circumstances—things that have been.> He paused, embarrassed.

As she replied, she raised her finger and pointed to her forehead:—

<I know. That is why I must go. I can tell you now, whilst the sun is coming up; I may not be able again. I know that when the Count wills me I must go. I know that if he tells me to come in secret, I must come by wile; by any device to hoodwink—even Jonathan.> God saw the look that she turned on me as she spoke, and if there be indeed a Recording Angel that look is noted to her everlasting honour. I could only clasp her hand. I could not speak; my emotion was too great for even the relief of tears. She went on:—

<You men are brave and strong. You are strong in your numbers, for you can defy that which would break down the human endurance of one who had to guard alone. Besides, I may be of service, since you can hypnotise me and so learn that which even I myself do not know.> Dr Van Helsing said very gravely:—

<Madam Mina, you are, as always, most wise. You shall with us come; and together we shall do that which we go forth to achieve.> When he had spoken, Mina's long spell of silence made me look at her. She had fallen back on her pillow asleep; she did not even wake when I had pulled up the blind and let in the sunlight which flooded the room. Van Helsing motioned to me to come with him quietly. We went to his room, and within a minute Lord Godalming, Dr Seward, and Mr Morris were with us also. He told them what Mina had said, and went on:—

<In the morning we shall leave for Varna. We have now to deal with a new factor: Madam Mina. Oh, but her soul is true. It is to her an agony to tell us so much as she has done; but it is most right, and we are warned in time. There must be no chance lost, and in Varna we must be ready to act the instant when that ship arrives.>

<What shall we do exactly?> asked Mr Morris laconically. The Professor paused before replying:—

<We shall at the first board that ship; then, when we have identified the box, we shall place a branch of the wild rose on it. This we shall fasten, for when it is there none can emerge; so at least says the superstition. And to superstition must we trust at the first; it was man's faith in the early, and it have its root in faith still. Then, when we get the opportunity that we seek, when none are near to see, we shall open the box, and—and all will be well.>

<I shall not wait for any opportunity,> said Morris. <When I see the box I shall open it and destroy the monster, though there were a thousand men looking on, and if I am to be wiped out for it the next moment!> I grasped his hand instinctively and found it as firm as a piece of steel. I think he understood my look; I hope he did.

<Good boy,> said Dr Van Helsing. <Brave boy. Quincey is all man. God bless him for it. My child, believe me none of us shall lag behind or pause from any fear. I do but say what we may do—what we must do. But, indeed, indeed we cannot say what we shall do. There are so many things which may happen, and their ways and their ends are so various that until the moment we may not say. We shall all be armed, in all ways; and when the time for the end has come, our effort shall not be lack. Now let us to-day put all our affairs in order. Let all things which touch on others dear to us, and who on us depend, be complete; for none of us can tell what, or when, or how, the end may be. As for me, my own affairs are regulate; and as I have nothing else to do, I shall go make arrangements for the travel. I shall have all tickets and so forth for our journey.>

There was nothing further to be said, and we parted. I shall now settle up all my affairs of earth, and be ready for whatever may come\ellipsispunct{.}
\end{diary}
 

\begin{diary}{Later.}
It is all done; my will is made, and all complete. Mina if she survive is my sole heir. If it should not be so, then the others who have been so good to us shall have remainder.

It is now drawing towards the sunset; Mina's uneasiness calls my attention to it. I am sure that there is something on her mind which the time of exact sunset will reveal. These occasions are becoming harrowing times for us all, for each sunrise and sunset opens up some new danger—some new pain, which, however, may in God's will be means to a good end. I write all these things in the diary since my darling must not hear them now; but if it may be that she can see them again, they shall be ready.

She is calling to me.

\end{diary}