%!TeX root=../draculatop.tex
\chapter[Chapter \thechapter]{}

\section{Jonathan Harker's Journal—continued}
	
When I found that I was a prisoner a sort of wild feeling came over me. I rushed up and down the stairs, trying every door and peering out of every window I could find; but after a little the conviction of my helplessness overpowered all other feelings. When I look back after a few hours I think I must have been mad for the time, for I behaved much as a rat does in a trap. When, however, the conviction had come to me that I was helpless I sat down quietly—as quietly as I have ever done anything in my life—and began to think over what was best to be done. I am thinking still, and as yet have come to no definite conclusion. Of one thing only am I certain; that it is no use making my ideas known to the Count. He knows well that I am imprisoned; and as he has done it himself, and has doubtless his own motives for it, he would only deceive me if I trusted him fully with the facts. So far as I can see, my only plan will be to keep my knowledge and my fears to myself, and my eyes open. I am, I know, either being deceived, like a baby, by my own fears, or else I am in desperate straits; and if the latter be so, I need, and shall need, all my brains to get through.

I had hardly come to this conclusion when I heard the great door below shut, and knew that the Count had returned. He did not come at once into the library, so I went cautiously to my own room and found him making the bed. This was odd, but only confirmed what I had all along thought—that there were no servants in the house. When later I saw him through the chink of the hinges of the door laying the table in the dining-room, I was assured of it; for if he does himself all these menial offices, surely it is proof that there is no one else to do them. This gave me a fright, for if there is no one else in the castle, it must have been the Count himself who was the driver of the coach that brought me here. This is a terrible thought; for if so, what does it mean that he could control the wolves, as he did, by only holding up his hand in silence. How was it that all the people at Bistritz and on the coach had some terrible fear for me? What meant the giving of the crucifix, of the garlic, of the wild rose, of the mountain ash? Bless that good, good woman who hung the crucifix round my neck! for it is a comfort and a strength to me whenever I touch it. It is odd that a thing which I have been taught to regard with disfavour and as idolatrous should in a time of loneliness and trouble be of help. Is it that there is something in the essence of the thing itself, or that it is a medium, a tangible help, in conveying memories of sympathy and comfort? Some time, if it may be, I must examine this matter and try to make up my mind about it. In the meantime I must find out all I can about Count Dracula, as it may help me to understand. To-night he may talk of himself, if I turn the conversation that way. I must be very careful, however, not to awake his suspicion.

 
\begin{diary}{Midnight.} 
I have had a long talk with the Count. I asked him a few questions on Transylvania history, and he warmed up to the subject wonderfully. In his speaking of things and people, and especially of battles, he spoke as if he had been present at them all. This he afterwards explained by saying that to a \textit{boyar} the pride of his house and name is his own pride, that their glory is his glory, that their fate is his fate. Whenever he spoke of his house he always said <we,> and spoke almost in the plural, like a king speaking. I wish I could put down all he said exactly as he said it, for to me it was most fascinating. It seemed to have in it a whole history of the country. He grew excited as he spoke, and walked about the room pulling his great white moustache and grasping anything on which he laid his hands as though he would crush it by main strength. One thing he said which I shall put down as nearly as I can; for it tells in its way the story of his race:—

<We Szekelys have a right to be proud, for in our veins flows the blood of many brave races who fought as the lion fights, for lordship. Here, in the whirlpool of European races, the Ugric tribe bore down from Iceland the fighting spirit which Thor and Wodin gave them, which their Berserkers displayed to such fell intent on the seaboards of Europe, ay, and of Asia and Africa too, till the peoples thought that the were-wolves themselves had come. Here, too, when they came, they found the Huns, whose warlike fury had swept the earth like a living flame, till the dying peoples held that in their veins ran the blood of those old witches, who, expelled from Scythia had mated with the devils in the desert. Fools, fools! What devil or what witch was ever so great as Attila, whose blood is in these veins?> He held up his arms. <Is it a wonder that we were a conquering race; that we were proud; that when the Magyar, the Lombard, the Avar, the Bulgar, or the Turk poured his thousands on our frontiers, we drove them back? Is it strange that when Arpad and his legions swept through the Hungarian fatherland he found us here when he reached the frontier; that the Honfoglalas was completed there? And when the Hungarian flood swept eastward, the Szekelys were claimed as kindred by the victorious Magyars, and to us for centuries was trusted the guarding of the frontier of Turkey-land; ay, and more than that, endless duty of the frontier guard, for, as the Turks say, <water sleeps, and enemy is sleepless.> Who more gladly than we throughout the Four Nations received the <bloody sword,> or at its warlike call flocked quicker to the standard of the King? When was redeemed that great shame of my nation, the shame of Cassova, when the flags of the Wallach and the Magyar went down beneath the Crescent? Who was it but one of my own race who as Voivode crossed the Danube and beat the Turk on his own ground? This was a Dracula indeed! Woe was it that his own unworthy brother, when he had fallen, sold his people to the Turk and brought the shame of slavery on them! Was it not this Dracula, indeed, who inspired that other of his race who in a later age again and again brought his forces over the great river into Turkey-land; who, when he was beaten back, came again, and again, and again, though he had to come alone from the bloody field where his troops were being slaughtered, since he knew that he alone could ultimately triumph! They said that he thought only of himself. Bah! what good are peasants without a leader? Where ends the war without a brain and heart to conduct it? Again, when, after the battle of Mohács, we threw off the Hungarian yoke, we of the Dracula blood were amongst their leaders, for our spirit would not brook that we were not free. Ah, young sir, the Szekelys—and the Dracula as their heart's blood, their brains, and their swords—can boast a record that mushroom growths like the Hapsburgs and the Romanoffs can never reach. The warlike days are over. Blood is too precious a thing in these days of dishonourable peace; and the glories of the great races are as a tale that is told.>

It was by this time close on morning, and we went to bed. (\textit{Mem.}, this diary seems horribly like the beginning of the \textit{Arabian Nights,} for everything has to break off at cockcrow—or like the ghost of Hamlet's father.)
\end{diary}
 
\begin{diary}{12 May.}
Let me begin with facts—bare, meagre facts, verified by books and figures, and of which there can be no doubt. I must not confuse them with experiences which will have to rest on my own observation, or my memory of them. Last evening when the Count came from his room he began by asking me questions on legal matters and on the doing of certain kinds of business. I had spent the day wearily over books, and, simply to keep my mind occupied, went over some of the matters I had been examining at Lincoln's Inn. There was a certain method in the Count's inquiries, so I shall try to put them down in sequence; the knowledge may somehow or some time be useful to me.

First, he asked if a man in England might have two solicitors or more. I told him he might have a dozen if he wished, but that it would not be wise to have more than one solicitor engaged in one transaction, as only one could act at a time, and that to change would be certain to militate against his interest. He seemed thoroughly to understand, and went on to ask if there would be any practical difficulty in having one man to attend, say, to banking, and another to look after shipping, in case local help were needed in a place far from the home of the banking solicitor. I asked him to explain more fully, so that I might not by any chance mislead him, so he said:—

<I shall illustrate. Your friend and mine, Mr Peter Hawkins, from under the shadow of your beautiful cathedral at Exeter, which is far from London, buys for me through your good self my place at London. Good! Now here let me say frankly, lest you should think it strange that I have sought the services of one so far off from London instead of some one resident there, that my motive was that no local interest might be served save my wish only; and as one of London residence might, perhaps, have some purpose of himself or friend to serve, I went thus afield to seek my agent, whose labours should be only to my interest. Now, suppose I, who have much of affairs, wish to ship goods, say, to Newcastle, or Durham, or Harwich, or Dover, might it not be that it could with more ease be done by consigning to one in these ports?> I answered that certainly it would be most easy, but that we solicitors had a system of agency one for the other, so that local work could be done locally on instruction from any solicitor, so that the client, simply placing himself in the hands of one man, could have his wishes carried out by him without further trouble.

<But,> said he, <I could be at liberty to direct myself. Is it not so?>

<Of course,> I replied; and <such is often done by men of business, who do not like the whole of their affairs to be known by any one person.>

<Good!> he said, and then went on to ask about the means of making consignments and the forms to be gone through, and of all sorts of difficulties which might arise, but by forethought could be guarded against. I explained all these things to him to the best of my ability, and he certainly left me under the impression that he would have made a wonderful solicitor, for there was nothing that he did not think of or foresee. For a man who was never in the country, and who did not evidently do much in the way of business, his knowledge and acumen were wonderful. When he had satisfied himself on these points of which he had spoken, and I had verified all as well as I could by the books available, he suddenly stood up and said:—

<Have you written since your first letter to our friend Mr Peter Hawkins, or to any other?> It was with some bitterness in my heart that I answered that I had not, that as yet I had not seen any opportunity of sending letters to anybody.

<Then write now, my young friend,> he said, laying a heavy hand on my shoulder: <write to our friend and to any other; and say, if it will please you, that you shall stay with me until a month from now.>

<Do you wish me to stay so long?> I asked, for my heart grew cold at the thought.

<I desire it much; nay, I will take no refusal. When your master, employer, what you will, engaged that someone should come on his behalf, it was understood that my needs only were to be consulted. I have not stinted. Is it not so?>

What could I do but bow acceptance? It was Mr Hawkins's interest, not mine, and I had to think of him, not myself; and besides, while Count Dracula was speaking, there was that in his eyes and in his bearing which made me remember that I was a prisoner, and that if I wished it I could have no choice. The Count saw his victory in my bow, and his mastery in the trouble of my face, for he began at once to use them, but in his own smooth, resistless way:—

<I pray you, my good young friend, that you will not discourse of things other than business in your letters. It will doubtless please your friends to know that you are well, and that you look forward to getting home to them. Is it not so?> As he spoke he handed me three sheets of note-paper and three envelopes. They were all of the thinnest foreign post, and looking at them, then at him, and noticing his quiet smile, with the sharp, canine teeth lying over the red underlip, I understood as well as if he had spoken that I should be careful what I wrote, for he would be able to read it. So I determined to write only formal notes now, but to write fully to Mr Hawkins in secret, and also to Mina, for to her I could write in shorthand, which would puzzle the Count, if he did see it. When I had written my two letters I sat quiet, reading a book whilst the Count wrote several notes, referring as he wrote them to some books on his table. Then he took up my two and placed them with his own, and put by his writing materials, after which, the instant the door had closed behind him, I leaned over and looked at the letters, which were face down on the table. I felt no compunction in doing so, for under the circumstances I felt that I should protect myself in every way I could.

One of the letters was directed to Samuel F\@. Billington, № 7, The Crescent, Whitby, another to Herr Leutner, Varna; the third was to Coutts \& Co., London, and the fourth to Herren Klopstock \& Billreuth, bankers, Buda-Pesth. The second and fourth were unsealed. I was just about to look at them when I saw the door-handle move. I sank back in my seat, having just had time to replace the letters as they had been and to resume my book before the Count, holding still another letter in his hand, entered the room. He took up the letters on the table and stamped them carefully, and then turning to me, said:—

<I trust you will forgive me, but I have much work to do in private this evening. You will, I hope, find all things as you wish.> At the door he turned, and after a moment's pause said:—

<Let me advise you, my dear young friend—nay, let me warn you with all seriousness, that should you leave these rooms you will not by any chance go to sleep in any other part of the castle. It is old, and has many memories, and there are bad dreams for those who sleep unwisely. Be warned! Should sleep now or ever overcome you, or be like to do, then haste to your own chamber or to these rooms, for your rest will then be safe. But if you be not careful in this respect, then>—He finished his speech in a gruesome way, for he motioned with his hands as if he were washing them. I quite understood; my only doubt was as to whether any dream could be more terrible than the unnatural, horrible net of gloom and mystery which seemed closing around me.
\end{diary}
 
\begin{diary}{Later.}
I endorse the last words written, but this time there is no doubt in question. I shall not fear to sleep in any place where he is not. I have placed the crucifix over the head of my bed—I imagine that my rest is thus freer from dreams; and there it shall remain.

When he left me I went to my room. After a little while, not hearing any sound, I came out and went up the stone stair to where I could look out towards the South. There was some sense of freedom in the vast expanse, inaccessible though it was to me, as compared with the narrow darkness of the courtyard. Looking out on this, I felt that I was indeed in prison, and I seemed to want a breath of fresh air, though it were of the night. I am beginning to feel this nocturnal existence tell on me. It is destroying my nerve. I start at my own shadow, and am full of all sorts of horrible imaginings. God knows that there is ground for my terrible fear in this accursed place! I looked out over the beautiful expanse, bathed in soft yellow moonlight till it was almost as light as day. In the soft light the distant hills became melted, and the shadows in the valleys and gorges of velvety blackness. The mere beauty seemed to cheer me; there was peace and comfort in every breath I drew. As I leaned from the window my eye was caught by something moving a storey below me, and somewhat to my left, where I imagined, from the order of the rooms, that the windows of the Count's own room would look out. The window at which I stood was tall and deep, stone-mullioned, and though weatherworn, was still complete; but it was evidently many a day since the case had been there. I drew back behind the stonework, and looked carefully out.

What I saw was the Count's head coming out from the window. I did not see the face, but I knew the man by the neck and the movement of his back and arms. In any case I could not mistake the hands which I had had so many opportunities of studying. I was at first interested and somewhat amused, for it is wonderful how small a matter will interest and amuse a man when he is a prisoner. But my very feelings changed to repulsion and terror when I saw the whole man slowly emerge from the window and begin to crawl down the castle wall over that dreadful abyss, \textit{face down} with his cloak spreading out around him like great wings. At first I could not believe my eyes. I thought it was some trick of the moonlight, some weird effect of shadow; but I kept looking, and it could be no delusion. I saw the fingers and toes grasp the corners of the stones, worn clear of the mortar by the stress of years, and by thus using every projection and inequality move downwards with considerable speed, just as a lizard moves along a wall.

What manner of man is this, or what manner of creature is it in the semblance of man? I feel the dread of this horrible place overpowering me; I am in fear—in awful fear—and there is no escape for me; I am encompassed about with terrors that I dare not think of\ellipsispunct{.}
\end{diary}
 
\begin{diary}{15 May.}
Once more have I seen the Count go out in his lizard fashion. He moved downwards in a sidelong way, some hundred feet down, and a good deal to the left. He vanished into some hole or window. When his head had disappeared, I leaned out to try and see more, but without avail—the distance was too great to allow a proper angle of sight. I knew he had left the castle now, and thought to use the opportunity to explore more than I had dared to do as yet. I went back to the room, and taking a lamp, tried all the doors. They were all locked, as I had expected, and the locks were comparatively new; but I went down the stone stairs to the hall where I had entered originally. I found I could pull back the bolts easily enough and unhook the great chains; but the door was locked, and the key was gone! That key must be in the Count's room; I must watch should his door be unlocked, so that I may get it and escape. I went on to make a thorough examination of the various stairs and passages, and to try the doors that opened from them. One or two small rooms near the hall were open, but there was nothing to see in them except old furniture, dusty with age and moth-eaten. At last, however, I found one door at the top of the stairway which, though it seemed to be locked, gave a little under pressure. I tried it harder, and found that it was not really locked, but that the resistance came from the fact that the hinges had fallen somewhat, and the heavy door rested on the floor. Here was an opportunity which I might not have again, so I exerted myself, and with many efforts forced it back so that I could enter. I was now in a wing of the castle further to the right than the rooms I knew and a storey lower down. From the windows I could see that the suite of rooms lay along to the south of the castle, the windows of the end room looking out both west and south. On the latter side, as well as to the former, there was a great precipice. The castle was built on the corner of a great rock, so that on three sides it was quite impregnable, and great windows were placed here where sling, or bow, or culverin could not reach, and consequently light and comfort, impossible to a position which had to be guarded, were secured. To the west was a great valley, and then, rising far away, great jagged mountain fastnesses, rising peak on peak, the sheer rock studded with mountain ash and thorn, whose roots clung in cracks and crevices and crannies of the stone. This was evidently the portion of the castle occupied by the ladies in bygone days, for the furniture had more air of comfort than any I had seen. The windows were curtainless, and the yellow moonlight, flooding in through the diamond panes, enabled one to see even colours, whilst it softened the wealth of dust which lay over all and disguised in some measure the ravages of time and the moth. My lamp seemed to be of little effect in the brilliant moonlight, but I was glad to have it with me, for there was a dread loneliness in the place which chilled my heart and made my nerves tremble. Still, it was better than living alone in the rooms which I had come to hate from the presence of the Count, and after trying a little to school my nerves, I found a soft quietude come over me. Here I am, sitting at a little oak table where in old times possibly some fair lady sat to pen, with much thought and many blushes, her ill-spelt love-letter, and writing in my diary in shorthand all that has happened since I closed it last. It is nineteenth century up-to-date with a vengeance. And yet, unless my senses deceive me, the old centuries had, and have, powers of their own which mere <modernity> cannot kill.
\end{diary}
 
\begin{diary}{Later: the Morning of 16 May.}
God preserve my sanity, for to this I am reduced. Safety and the assurance of safety are things of the past. Whilst I live on here there is but one thing to hope for, that I may not go mad, if, indeed, I be not mad already. If I be sane, then surely it is maddening to think that of all the foul things that lurk in this hateful place the Count is the least dreadful to me; that to him alone I can look for safety, even though this be only whilst I can serve his purpose. Great God! merciful God! Let me be calm, for out of that way lies madness indeed. I begin to get new lights on certain things which have puzzled me. Up to now I never quite knew what Shakespeare meant when he made Hamlet say:—

\begin{quote}My tablets! quick, my tablets!\\
'Tis meet that I put it down,
\end{quote} etc., for now, feeling as though my own brain were unhinged or as if the shock had come which must end in its undoing, I turn to my diary for repose. The habit of entering accurately must help to soothe me.

The Count's mysterious warning frightened me at the time; it frightens me more now when I think of it, for in future he has a fearful hold upon me. I shall fear to doubt what he may say!

When I had written in my diary and had fortunately replaced the book and pen in my pocket I felt sleepy. The Count's warning came into my mind, but I took a pleasure in disobeying it. The sense of sleep was upon me, and with it the obstinacy which sleep brings as outrider. The soft moonlight soothed, and the wide expanse without gave a sense of freedom which refreshed me. I determined not to return to-night to the gloom-haunted rooms, but to sleep here, where, of old, ladies had sat and sung and lived sweet lives whilst their gentle breasts were sad for their menfolk away in the midst of remorseless wars. I drew a great couch out of its place near the corner, so that as I lay, I could look at the lovely view to east and south, and unthinking of and uncaring for the dust, composed myself for sleep. I suppose I must have fallen asleep; I hope so, but I fear, for all that followed was startlingly real—so real that now sitting here in the broad, full sunlight of the morning, I cannot in the least believe that it was all sleep.

I was not alone. The room was the same, unchanged in any way since I came into it; I could see along the floor, in the brilliant moonlight, my own footsteps marked where I had disturbed the long accumulation of dust. In the moonlight opposite me were three young women, ladies by their dress and manner. I thought at the time that I must be dreaming when I saw them, for, though the moonlight was behind them, they threw no shadow on the floor. They came close to me, and looked at me for some time, and then whispered together. Two were dark, and had high aquiline noses, like the Count, and great dark, piercing eyes that seemed to be almost red when contrasted with the pale yellow moon. The other was fair, as fair as can be, with great wavy masses of golden hair and eyes like pale sapphires. I seemed somehow to know her face, and to know it in connection with some dreamy fear, but I could not recollect at the moment how or where. All three had brilliant white teeth that shone like pearls against the ruby of their voluptuous lips. There was something about them that made me uneasy, some longing and at the same time some deadly fear. I felt in my heart a wicked, burning desire that they would kiss me with those red lips. It is not good to note this down, lest some day it should meet Mina's eyes and cause her pain; but it is the truth. They whispered together, and then they all three laughed—such a silvery, musical laugh, but as hard as though the sound never could have come through the softness of human lips. It was like the intolerable, tingling sweetness of water-glasses when played on by a cunning hand. The fair girl shook her head coquettishly, and the other two urged her on. One said:—

<Go on! You are first, and we shall follow; yours is the right to begin.> The other added:—

<He is young and strong; there are kisses for us all.> I lay quiet, looking out under my eyelashes in an agony of delightful anticipation. The fair girl advanced and bent over me till I could feel the movement of her breath upon me. Sweet it was in one sense, honey-sweet, and sent the same tingling through the nerves as her voice, but with a bitter underlying the sweet, a bitter offensiveness, as one smells in blood.

I was afraid to raise my eyelids, but looked out and saw perfectly under the lashes. The girl went on her knees, and bent over me, simply gloating. There was a deliberate voluptuousness which was both thrilling and repulsive, and as she arched her neck she actually licked her lips like an animal, till I could see in the moonlight the moisture shining on the scarlet lips and on the red tongue as it lapped the white sharp teeth. Lower and lower went her head as the lips went below the range of my mouth and chin and seemed about to fasten on my throat. Then she paused, and I could hear the churning sound of her tongue as it licked her teeth and lips, and could feel the hot breath on my neck. Then the skin of my throat began to tingle as one's flesh does when the hand that is to tickle it approaches nearer—nearer. I could feel the soft, shivering touch of the lips on the super-sensitive skin of my throat, and the hard dents of two sharp teeth, just touching and pausing there. I closed my eyes in a languorous ecstasy and waited—waited with beating heart.

But at that instant, another sensation swept through me as quick as lightning. I was conscious of the presence of the Count, and of his being as if lapped in a storm of fury. As my eyes opened involuntarily I saw his strong hand grasp the slender neck of the fair woman and with giant's power draw it back, the blue eyes transformed with fury, the white teeth champing with rage, and the fair cheeks blazing red with passion. But the Count! Never did I imagine such wrath and fury, even to the demons of the pit. His eyes were positively blazing. The red light in them was lurid, as if the flames of hell-fire blazed behind them. His face was deathly pale, and the lines of it were hard like drawn wires; the thick eyebrows that met over the nose now seemed like a heaving bar of white-hot metal. With a fierce sweep of his arm, he hurled the woman from him, and then motioned to the others, as though he were beating them back; it was the same imperious gesture that I had seen used to the wolves. In a voice which, though low and almost in a whisper seemed to cut through the air and then ring round the room he said:—

<How dare you touch him, any of you? How dare you cast eyes on him when I had forbidden it? Back, I tell you all! This man belongs to me! Beware how you meddle with him, or you'll have to deal with me.> The fair girl, with a laugh of ribald coquetry, turned to answer him:—

<You yourself never loved; you never love!> On this the other women joined, and such a mirthless, hard, soulless laughter rang through the room that it almost made me faint to hear; it seemed like the pleasure of fiends. Then the Count turned, after looking at my face attentively, and said in a soft whisper:—

<Yes, I too can love; you yourselves can tell it from the past. Is it not so? Well, now I promise you that when I am done with him you shall kiss him at your will. Now go! go! I must awaken him, for there is work to be done.>

<Are we to have nothing to-night?> said one of them, with a low laugh, as she pointed to the bag which he had thrown upon the floor, and which moved as though there were some living thing within it. For answer he nodded his head. One of the women jumped forward and opened it. If my ears did not deceive me there was a gasp and a low wail, as of a half-smothered child. The women closed round, whilst I was aghast with horror; but as I looked they disappeared, and with them the dreadful bag. There was no door near them, and they could not have passed me without my noticing. They simply seemed to fade into the rays of the moonlight and pass out through the window, for I could see outside the dim, shadowy forms for a moment before they entirely faded away.

Then the horror overcame me, and I sank down unconscious.
\end{diary}